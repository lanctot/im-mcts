
\documentclass{article}

%\setlength{\topmargin}{-0.75in}
\setlength{\oddsidemargin}{0.3in}
\setlength{\evensidemargin}{0.3in}
\setlength{\textwidth}{6in}
%\setlength{\textheight}{9.5in}

\usepackage{graphicx}
\usepackage{amssymb}
\usepackage{amsmath}
\usepackage{amssymb}
\usepackage{color}
\usepackage{tikz}
\usepackage{pgfplots}
\usepackage{caption}
\usepackage{subcaption}
\usepackage{comment}
\usepackage{booktabs}
\usepackage{nopageno}
\usepackage{url}

\usepackage[algo2e, noend, noline, linesnumbered]{algorithm2e}
\DontPrintSemicolon

\makeatletter
\newcommand{\pushline}{\Indp}% Indent
\newcommand{\popline}{\Indm}
\makeatother
\newcommand{\argmax}{\operatornamewithlimits{argmax}}

\captionsetup{compatibility=false}



%\pgfplotsset{compat=newest}
\usetikzlibrary{arrows,shapes,petri}

\newcommand{\bE}{\mathbb{E}}
\newcommand{\cA}{\mathcal{A}}
\newcommand{\cC}{\mathcal{C}}
\newcommand{\cD}{\mathcal{D}}
\newcommand{\cI}{\mathcal{I}}
\newcommand{\cN}{\mathcal{N}}
\newcommand{\cO}{\mathcal{O}}
\newcommand{\cS}{\mathcal{S}}
\newcommand{\cT}{\mathcal{T}}
\newcommand{\cZ}{\mathcal{Z}}
\newcommand{\eg}{{\it e.g.,}~}
\newcommand{\ie}{{\it i.e.,}~}

\newcommand{\redbold}[1]{\textbf{\color{red}#1}} 
\newcommand{\markw}[1]{\textbf{\color{red} /* #1 (markw) */}} 
\newcommand{\marcl}[1]{\textbf{\color{red} /* #1 (marcl) */}} 

\begin{document}

\title{Improving Monte Carlo Tree Search with Heuristic Evaluations using Implicit Minimax Backups}

\author{Authors}

\maketitle

\begin{abstract}
In this document we describe how minimax backups from evaluations functions can be combined implicitly 
with MCTS. 
\end{abstract}

\section{Introduction}

Monte Carlo Tree Search (MCTS)~\cite{Coulom06Efficient,Kocsis06Bandit} 

\section{Background and Related Work} 

Several techniques for minimax-style backup rules in the simulation-based MCTS framework have already been proposed. 
The first was Coulom's original {\it maximum backpropagation}~\cite{Coulom06Efficient}. This method of backpropagation
suggests, after a number of simulations to a node has been reached, to switch to propagating the maximum value instead 
of the simulated (``average'') value. 
The rationale behind this choice is that after a certain point, the search algorithm should consider the node
{\it converged} and return an estimate of the best value. 
Maximum backpropagation focuses on estimating the value of the optimal action required for the parent nodes, which is a
``separate exploratory concern'' than simply identifying the best action at that node~\cite{Feldman13Theoretically}.
In games where minimax search works well, such as Mancala, modifying MCTS to 
use minimax-style backups and heuristic values instead of playouts was shown to be as powerful as a strong 
depth-first minimax search~\cite{Ramanujan11Tradeoffs}.
Similarly, there is further evidence suggesting not replacing the playout entirely, but terminating them early 
using heuristic evaluations, has increased the performance in Lines of Action~\cite{bla}, Amazons~\cite{bla}, and 
Breakthrough~\cite{bla}. 

% mention score-bounded mcts
% mention hendrik's minimax methods
% use mcts-solver to motivate implicit minimax

One way to combine minimax backups implictly is to use MCTS-Solver~\cite{Winands08Solver}. When using 
MCTS-Solver, proven wins and losses are backpropagated as extra information in MCTS. When a node is proven to be a 
win or a loss, it no longer needs to be searched. This simple, domain-independent modification greatly enhances 
MCTS in many games (some examples). 
\marcl{Also mention MCTS + minimax hybrids (Baier \& Winands) and Score-bounded MCTS (...))}

In this work, we propose to augment MCTS with the help of an implictly-computed minimax search which uses
heuristic evaluations. In our case, these heurstic evaluations are not used to replace or terminate playouts, but 
rather as a {\it separate source of information} to guide the tree search during the selection phase. Like MCTS-Solver, 
the backpropagation of these values is treated differently and affect node selection. However, unlike MCTS-Solver and 
minimax hybrids, these modifications are based on heuristic evaluations rather than proven wins and losses. 
We show that (something good and insightful). 

\bibliographystyle{unsrt}
\bibliography{im-mcts}

\end{document}
