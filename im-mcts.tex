%File: formatting-instruction.tex
\documentclass[letterpaper]{article}
\usepackage{aaai}
\usepackage{times}
\usepackage{helvet}
\usepackage{courier}
\usepackage{graphicx}
\usepackage{amssymb}
\usepackage{amsmath}
\usepackage{amssymb}
\usepackage{color}
\usepackage{tikz}
\usepackage{pgfplots}
\usepackage{caption}
\usepackage{subcaption}
\usepackage{comment}
\usepackage{booktabs}
\usepackage{nopageno}
\usepackage{url}

\usepackage[algo2e, noend, noline, linesnumbered]{algorithm2e}
\providecommand{\SetAlgoLined}{\SetLine}  
\providecommand{\DontPrintSemicolon}{\dontprintsemicolon}
\DontPrintSemicolon
\makeatletter
\newcommand{\pushline}{\Indp}% Indent
\newcommand{\popline}{\Indm}
\makeatother
\newcommand{\argmax}{\operatornamewithlimits{argmax}}

\captionsetup{compatibility=false}

%\pgfplotsset{compat=newest}
\usetikzlibrary{arrows,shapes,petri}

\newcommand{\bE}{\mathbb{E}}
\newcommand{\bQ}{\bar{Q}}
\newcommand{\cA}{\mathcal{A}}
\newcommand{\cC}{\mathcal{C}}
\newcommand{\cD}{\mathcal{D}}
\newcommand{\cG}{\mathcal{G}}
\newcommand{\cI}{\mathcal{I}}
\newcommand{\cN}{\mathcal{N}}
\newcommand{\cO}{\mathcal{O}}
\newcommand{\cR}{\mathcal{R}}
\newcommand{\cS}{\mathcal{S}}
\newcommand{\cT}{\mathcal{T}}
\newcommand{\cZ}{\mathcal{Z}}
\newcommand{\hQ}{\hat{Q}}
\newcommand{\hV}{\hat{V}}
\newcommand{\eg}{{\it e.g.,}~}
\newcommand{\ie}{{\it i.e.,}~}

% the note center!
\definecolor{darkgreen}{RGB}{0,125,0}
\newcounter{nsNoteCounter}
\newcounter{mlNoteCounter}
\newcounter{mwNoteCounter}
\newcounter{tpNoteCounter}
\newcommand{\nsnote}[1]{{\scriptsize \color{blue} $\blacksquare$ \refstepcounter{vlNoteCounter}\textsf{[NS]$_{\arabic{vlNoteCounter}}$:{#1}}}}
\newcommand{\mlnote}[1]{{\scriptsize \color{darkgreen} $\blacksquare$ \refstepcounter{mlNoteCounter}\textsf{[ML]$_{\arabic{mlNoteCounter}}$:{#1}}}}
\newcommand{\mwnote}[1]{{\scriptsize \color{red} $\blacktriangle$ \refstepcounter{asNoteCounter}\textsf{[AS]$_{\arabic{asNoteCounter}}$:{#1}}}}
\newcommand{\tpnote}[1]{{\scriptsize \color{orange} $\blacktriangle$ \refstepcounter{asNoteCounter}\textsf{[TP]$_{\arabic{asNoteCounter}}$:{#1}}}}
%\newcounter{NoteCounter}
%\newcommand{\vlnote}[1]{{\scriptsize \color{blue} \refstepcounter{vlNoteCounter}\textsf{[VL]$_{\arabic{vlNoteCounter}}$:{#1}}}}
%\renewcommand{\vlnote}[1]{}


\frenchspacing
\setlength{\pdfpagewidth}{8.5in}
\setlength{\pdfpageheight}{11in}
\pdfinfo{
/Title (Insert Your Title Here)
/Author (Put All Your Authors Here, Separated by Commas)}
\setcounter{secnumdepth}{0}  

\begin{document}
% The file aaai.sty is the style file for AAAI Press 
% proceedings, working notes, and technical reports.
%
\title{Monte Carlo Tree Search with Heuristic Evaluations\\using Implicit Minimax Backups}
\author{Authors}

\maketitle

\begin{abstract}
Monte Carlo Tree Search (MCTS) has improved the performance of game-playing engines in 
domains such as Go, Hex, and general-game playing. MCTS has been shown to outperform 
classic minimax search in games where good heuristic evaluations are difficult to obtain. 
In recent years, combining 
ideas from traditional minimax search in MCTS has been shown to be advantageous in some domains,
such as Lines of Action, Mancala, Amazons, Breakthrough, and Connect Four. In this paper, we propose 
a new way to use heuristic evaluations to guide the MCTS search by storing the two sources of 
information, estimated win rates and heuristic evaluations, separately. 
Rather than using the heuristic evaluations to replace the playouts, 
our technique backs them up {\it implicity} during its MCTS simulations. 
These learned evaluation values are then used to guide future simulations. 
Compared to current techniques, we show that using implicit minimax backups  
reduces regret in an example game and leads to stronger 
play performance in four games: Lines of Action, Chinese Checkers, Mancala, and Breakthrough. 
\end{abstract}

%\subsection{Related Work}

\section{Introduction}

Monte Carlo Tree Search (MCTS)~\cite{Coulom06Efficient,Kocsis06Bandit} is a simulation-based best-first
search paradigm that has been shown to increase performane in domains such as turn-taking games, 
general-game playing, real-time strategy games, single-agent planning, and more~\cite{mctssurvey}. 
While the initial applications have been to games where heursitic evaluations are difficult to obtain, 
progress in MCTS research has shown that heuristics can be effectively be combined in MCTS, even in games 
where class minimax search has traditonally been preferred. 

The most popular Monte Carlo tree search algorithm is UCT~\cite{Kocsis06Bandit}, an algorithm
which performs a single simulation from the root of the search tree to a terminal state at each iteration. 
During the iterative process, a model of the game tree is incrementally built by adding a 
new leaf node to the tree on each iteration. 
The nodes in the tree are used to store statistical information regarding wins and losses, and backpropagation
policies are used to update parent estimates, and these improving estimates are then used to guide future simulations. 
When the simulation reaches parts of the game tree not included in the model, a default playout policy
is used to simulate to a terminal state where a win or loss is determined. 

%When applied to sequential turn-taking games, MCTS has primarily been used in domains where good 
%heuristic evaluations are difficult to obtain. 
In this work, we propose to augment MCTS with the help of 
an implicitly-computed minimax search which uses heuristic evaluations. 
Unlike previous work, these heuristic evaluations are used as {\it separate source of information}, 
and backed up in the same way as in classic minimax search. These minimax-style backups are done {\it implicitly}
as a simple extra step during the standard simulation, and always maintained separately from win rate estimates obtained
from playouts. These two separate information sources are then used to guide the MCTS search. 
We show that combining heuristic evaluations in this way reduces simple regret and improves performance in four 
different games. 

\subsection{Related Work}

Several techniques for minimax-influenced backup rules in the simulation-based MCTS framework have already been proposed. 
The first was Coulom's original {\it maximum backpropagation}~\cite{Coulom06Efficient}. This method of backpropagation
suggests, after a number of simulations to a node has been reached, to switch to propagating the maximum value instead 
of the simulated (``average'') value. 
The rationale behind this choice is that after a certain point, the search algorithm should consider the node
{\it converged} and return an estimate of the best value. 
Maximum backpropagation has also been used in other Monte Carlo tree search algorithms and demonstrated success in
probabilistic planning, as an alternative type of forecaster in BRUE~\cite{Feldman13Theoretically} and as Bellman 
backups for online dynamic programming in Trial-based Heuristic Tree Search~\cite{Keller13Trial}.

The first use of enhancing MCTS using prior knowledge was in Computer Go~\cite{Gelly07Combining}. 
In this work, offline-learned knowledge initialized values of expanded nodes increased performance against significantly against 
strong benchmark player. This technique was also confirmed to be advantageous in Breakthrough~\cite{Lorentz13Breakthrough}. 
Another way to introduce prior knowledge is a progressive bias during selection~\cite{Chaslot08Progressive}, which also showed 
increased performance in Go play strength. 

In games where minimax search performs well, such as Mancala, modifying MCTS to 
use minimax-style backups and heuristic values instead to replace playouts offers a worthwhile trade-off 
under different search time settings~\cite{Ramanujan11Tradeoffs}.
Similarly, there is further evidence suggesting not replacing the playout entirely, but terminating them early 
using heuristic evaluations, has increased the performance in Lines of Action (LOA)~\cite{Winands10MCTS-LOA}, 
Amazons~\cite{Kloetzer10Amazons,Lorentz08Amazons}, and Breakthrough~\cite{Lorentz13Breakthrough}. In LOA and Amazons, the 
MCTS players enhanced with evaluation functions outperform their minimax counterparts using the same evaluation function.

% mention score-bounded mcts
% mention hendrik's minimax methods
% use mcts-solver to motivate implicit minimax
%\marcl{Also mention MCTS + minimax hybrids (Baier \& Winands) and Score-bounded MCTS (...))}

One may want to combine minimax backups or searches without using an evaluation function. 
The prime example is MCTS-Solver~\cite{Winands08Solver}. When using 
MCTS-Solver, proven wins and losses are backpropagated as extra information in MCTS. When a node is proven to be a 
win or a loss, it no longer needs to be searched. This simple, domain-independent modification greatly enhances 
MCTS in many games (some examples). Score-bounded MCTS extends this idea to games with multiple outcomes, 
leading to $\alpha \beta$ style pruning in the tree~\cite{Cazenave10ScoreBounded}. Finally, one can use hybrid
minimax searches in the tree to initialize nodes during, enhance the playout, or to help MCTS-Solver 
in backpropagation~\cite{Baier13MinimaxHybrids}.

Finally, recent work has attempted to explain and identify some of the shortcomings that arise from estimates in 
MCTS, specifically compared to situations where classic minimax search has historically performed 
well~\cite{Ramanujan10Understanding,Ramanujan10On}. 
Attempts have been made to overcome the problem of {\it traps} or {\it optimistic moves} (strategic  
tactial lines that are problematic for MCTS) such as sufficiency thresholds during selection~\cite{Gudmindsson13Sufficiency} 
and shallow minimax searches~\cite{Baier13MinimaxHybrids}. We show that implicit minimax backups also
improve performance in domains with tactical short-term goals. 

\section{Adversarial Search in Turn-based Games}

%The following notation is taken from~\cite{Sutton98}

A finite deterministic Markov Decision Process (MDP) is 4-tuple $(\cS, \cA, \cT, \cR)$. Here, $\cS$ is a finite non-empty set of {\it states}. 
$\cA$ is a finite non-empty set of {\it actions}, where we denote $\cA(s) \subseteq \cA$ the set of available actions at state $s$. 
$\cT : \cS \times \cA \mapsto \Delta \cS$ is a {\it transition function} mapping 
each state and action to a distribution over successor states. Finally, $\cR : \cS \times \cA \times \cS \mapsto \Re$ 
is a {\it reward function} mapping (state, action, successor state) triplets to numerical rewards. 

A two-player perfect information game is an MDP with a specific form. 
Denote $\cZ = \{ s \in \cS: \cA(s) = \emptyset \} \subset \cS$ the set of {\it terminal states}. 
In addition, for all nonterminal states $s' \in \cS - \cZ$, $\cR(s,a,s') = 0$. 
There is a {\it player identity function} $\tau : \cS - \cZ \mapsto \{1,2\}$. 
In this paper, we assume fully deterministic domains, so $\cT(s,a)$ maps to a single state. 
When it is clear from the context and unless otherwise stated, de denote $s' = \cT(s,a)$. 
However, the ideas proposed can be easily extended to domains with stochastic transitions. 

Monte Carlo Tree Search is a simulation-based best-first search algorithm that incrementally builds a model of the game 
$\cG$, in memory. Each search starts with from a {\it root state} $s_0 \in \cS - \cZ$, and initially sets $\cG = \emptyset$. 
Each simulation samples a trajectory $\rho = (s_0, a_0, s_1, a_1, \cdots, s_n)$, where $s_n \in \cZ$. 
The portion of the $\rho$ where $s_i \in \cG$ is called the {\it tree portion} and the remaining portion is
called the {\it playout portion}. In the tree portion, actions are chosen according to some {\it selection policy}. 
The first state encountered in the playout portion is {\it expanded} (added to $\cG$.) 
The actions chosen in the playout portion are determined by a specific {\it playout policy}. 
States $s \in \cG$ are referred to as {\it nodes} and statistics are  
maintained for each node $s$: the cumulative reward, $r_s$, and visit count, $n_s$. 
By popular convention, we define $r_{s,a} = r_{s'}$ where $s' = \cT(s,a)$, and similarly $n_{s,a} = n_{s'}$. 
Also, we use $r^{\tau}_s$ to denote the reward at state $s$ {\it with respect to player} $\tau(s)$. 
%Finally, we denote  the set $C(s)$ to be the children of $s$ that are in $\cG$. 

Let $\hV(s,a)$ be an estimator for node $s$ and $\hQ(s,a)$ for the state-action pair. 
For example, one popular estimator is the observed mean over all simulations 
$\bQ(s,a) = r^{\tau}_{s,a} / n_{s,a}$. 
The most widely-used is based on a bandit algorithm called Upper Confidence Bounds (UCB)~\cite{Auer02Finite} in an algorithm adapted 
for MCTS called UCT~\cite{Kocsis06Bandit}, which selects action $a'$ using
\begin{equation}
\label{eq:select-ucb}
a' = \argmax_{a \in \cA(s)} \left\{ \hQ(s,a) + C \sqrt{\frac{\ln n_s}{n_{s,a}}} \right\}, 
\end{equation}
where $C$ is parameter determining the weight of exploration. 

\section{Implicit Minimax Backups in MCTS}

Now, suppose we are given an evaluation function $v_0(s)$ whose range is the same as that of the reward function $\cR$. 
Assuming $v_0(s)$ is a sensible indicator of the reward, we would like that this added source of information
strictly benefits MCTS. 
To make use of this information in MCTS, we add another value to maintain at each node, the 
{\it implicit minimax evaluation with respect to player} $\tau(s)$, $v^{\tau}_s$, and define $v^{\tau}_{s,a}$ as before. 
During backpropagation, $r_s$ and $n_s$ are updated in the usual way, and additionally $v^{\tau}_s$ is updated a minimax backup 
rule based on children values. Also, rather than using $\hQ = \bQ$ for selection in Equation~\ref{eq:select-ucb}, we use
\begin{equation}
\label{eq:imq}
\hQ^{\mathit{IM}}(s,a) = (1-\alpha) \frac{r^{\tau}_{s,a}}{n_{s,a}} + \alpha v^{\tau}_{s,a}.
\end{equation}

Pseudo-code is presented in Algorithm~\ref{alg}. There are three main changes from vanilla MCTS, two 
of which are on lines \ref{alg:imselect} and \ref{alg:imeval}. 
During selection, $\hQ^{\mathit{IM}}$ from Equation~\ref{eq:imq} replaces $\bQ$ in 
Equation~\ref{eq:select-ucb}. During backpropagation, the implicit minimax evaluations $v^{\tau}_s$ are updated based on 
the children's values. Note that a single $\max$ is used here since the evaluations are assumed to be in 
view of player $\tau(s)$.
%\footnote{Domain-dependent inversions or scalings may be necessary to convert the rewards. These details are 
%intentionally omitted from the pseudo-code for generality.}
The function $\alpha(n_s)$ will determine how much weight to attribute to these evaluations.  
Finally, during a node expansion on line~\ref{alg:imexpand}, the implicit minimax value is initialized to its heuristic
evaluation $v^{\tau}_s \leftarrow v^{\tau}_0(s)$. 

\newcommand{\MCTS}{{\sc MCTS}}
\newcommand{\Update}{{\sc Update}}
\newcommand{\Playout}{{\sc Playout}}
\newcommand{\Select}{{\sc Select}}
\newcommand{\Expand}{{\sc Expand}}
\newcommand{\Simulate}{{\sc Simulate}}

\begin{algorithm2e}[h!]
  \Select$(s)$:\;
  \pushline
    Let $A'$ be the set of actions $a \in \cA(s)$ maximizing $\hQ^{\mathit{IM}}(s,a) + C \sqrt{\frac{\ln n_s}{n_{s,a}}}$ \; \label{alg:imselect}
    {\bf return} $a' \sim ${\sc Uniform}$(A')$ \;
  \popline
  \;
  \Update$(s,r)$:\;
  \pushline
    $r_s \leftarrow r_s + r$\;
    $n_s \leftarrow n_s + 1$\;
    $v^{\tau}_s \leftarrow \max_{a \in \cA(s)} v^{\tau}_{s,a}$\; \label{alg:imeval}
  \popline
  \;
  \Simulate$(s_{prev}, a_{prev}, s)$:\;
  \pushline
    \If{$s \not\in \cG$}{
      \Expand($s$)\;                                           \label{alg:imexpand}
      $r \leftarrow $\Playout($s$)\;
      \Update($s,r$)\;
      {\bf return} $r$\;
    }
    \Else{ 
      \lIf{$s \in \cZ$}{{\bf return} $\cR(s_{prev}, a_{prev}, s)$}
      $a \leftarrow $\Select$(s)$\;
      $s' \leftarrow \cT(s,a)$\;
      $r \leftarrow $\Simulate$(s,a,s')$\;
      \Update($s,r$)\;
      {\bf return} $r$\;
    }
  \popline
  \;
  \MCTS($s_0$):\;
  \pushline
    \lWhile{\mbox{time left}}{\Simulate$(-,-,s_0)$}
    {\bf return} $\argmax_{a \in \cA(s_0)} n_{s_0,a}$\; 
  \popline
  \vspace{0.3cm}
  \caption{Pseudo-code of MCTS with Implicit Minimax Backups. \label{alg}}
\end{algorithm2e}

In essence, MCTS with implicit minimax backups acts like a heuristic approximation of MCTS-Solver for the portion of the
search tree that has not reached terminal states. 
However, unlike MCTS-Solver and minimax hybrids, these modifications are based on heuristic evaluations rather than 
proven wins and losses. 

\section{Empirical Evaluation}

\newcommand{\UCTMAXH}{$\mbox{UCTMAX}_H$\xspace}
\newcommand{\UCTH}{$\mbox{UCT}_H$\xspace}

Some important things we might want to consider for the results:

\begin{itemize}
\item {\bf Early playout terminations}. When a playout is started from state $s$, descend $d$ plies by 
      choosing actions according to a playout policy to reach state $s^d$, then return the value 
      of the $v^{\tau}_0(s^d)$. In Mancala, \UCTH uses early terminations with $d = 0$. 
      In previous work in Amazons and Breakthrough, and some domains with chance, 
      best performance is achieved for $d > 0$. 
\item {\bf $\epsilon$-greedy playouts}. 
      At state $s$, the playout policy chooses randomly with 
      probability $\epsilon$ and chooses an action $a \in \cA(s)$ that maximizes $v_0^\tau(\cT(s,a))$. 
\item {\bf Constant/Progressive bias.} Constant or progressive bias defines $\hQ$ as in Equation \ref{eq:imq} 
      except replaces $v^{\tau}_{s,a}$ with $v^{\tau}_0(s')$ or with $v^{\tau}_0(s') / (n_{s,a} + 1)$, respectively. 
\item {\bf Node priors} (see Lorentz paper, cite previous papers on this)
\item Probably the most similar competitor is \UCTMAXH from \cite{Ramanujan11Tradeoffs}. \UCTH
      is standard UCT with early playout terminations ($d = 0$). \UCTMAXH is the same, except maximum backpropagation
      is used. Note that \UCTMAXH is different from implicit minimax; implicit minimax keeps the average computed by 
      the playouts and minimax values separately, benefiting from both sources of information. I expect implicit minimax
      to work better than \UCTMAXH, but it might be nice to show this (esp. in Mancala). 
\item The extension to games with chance nodes is straight-forward. It might be nice to show it, though. There is a tactical 
      candidate game growing in popularity (Chinese Dark Chess). 
\item Simple regret / observed error? We need a small game that's solveable so we can compute the optimal minimax values.
\end{itemize}

Candidate domains at the moment: Breakthrough, Cannon, Chinese Checkers, Lines of Action, Mancala, Chinese Dark Chess, Hearts.
In terms of choosing domains, it seems like this work on ``somewhat tactical games'' (where minimaxing will help). 

A comparison to standard minimax seems appropriate. It'd be nice to find a game where MCTS with implicit minimax backups
is preferred over all of the others (minimax, MCTS, \UCTMAXH). 


\bibliographystyle{aaai}
\bibliography{im-mcts}


\end{document}
