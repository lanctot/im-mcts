
\documentclass{article}

% For figures
\usepackage{graphicx} % more modern
%\usepackage{epsfig} % less modern
\usepackage{subfigure} 

% For citations
\usepackage{natbib}
%\usepackage{named}

% For algorithms
\usepackage{algorithm}
\usepackage{algorithmic}

% As of 2011, we use the hyperref package to produce hyperlinks in the
% resulting PDF.  If this breaks your system, please commend out the
% following usepackage line and replace \usepackage{icml2012} with
% \usepackage[nohyperref]{icml2012} above.
\usepackage{hyperref}

% Packages hyperref and algorithmic misbehave sometimes.  We can fix
% this with the following command.
\newcommand{\theHalgorithm}{\arabic{algorithm}}

% Employ the following version of the ``usepackage'' statement for
% submitting the draft version of the paper for review.  This will set
% the note in the first column to ``Under review.  Do not distribute.''
% Employ this version of the ``usepackage'' statement after the paper has
% been accepted, when creating the final version.  This will set the
% note in the first column to ``Appearing in''
% \usepackage[accepted]{icml2012}

\usepackage{times}
\usepackage{helvet}
\usepackage{courier}
\usepackage{color}
\usepackage{colortbl}
\usepackage{algorithm}
\usepackage{algorithmic}
\usepackage{amsmath}
\usepackage{amssymb}
\usepackage{latexsym} 
\usepackage{multirow}
\usepackage{verbatim}
\usepackage{amsthm}

\usepackage[margin=1in]{geometry}

\newcommand{\II}{\mathcal{I}}
\newcommand{\PP}{\check{\mathcal{P}}}
\newcommand{\RR}{\mathbb{R}}
\newcommand{\EV}{\mathbb{E}}
\newcommand{\cE}{\mathcal{E}}
\newcommand{\VAR}{\mathbb{V}}
\newcommand{\abs}[1]{\left|#1\right|}
\newcommand{\Qed}{$\blacksquare$}
\newtheorem{definition}{Definition}
\newtheorem{fact}{Fact}
\newtheorem{theorem}{Theorem}
\newtheorem{corollary}{Corollary}
\newtheorem{condition}{Condition}
%\newenvironment{proof}{\begin{trivlist}\item[\hspace{\labelsep}{\bf\noindent Proof: }]}{\hspace{\stretch{1}}\rule{1ex}{1ex}\end{trivlist}}
\newcommand{\defword}[1]{\textbf{\boldmath{#1}}}
\newcommand{\citejustyear}[1]{\cite{#1}}

\def\etal{\textit{et al.}}
\def\ie{\textit{i.e.}}
\def\eg{\textit{e.g.}}

\newtheorem{df}{Definition}
\newtheorem*{definition4}{Definition 4}
\newtheorem{notation}{Notation}
\newtheorem*{theorem2}{Theorem 2}
\newtheorem*{theorem3}{Theorem 3}
\newtheorem{theorem1}{Theorem}
\newtheorem*{lemmaA}{Lemma A}
\newtheorem{lemma}{Lemma}[section]
\newtheorem{col}{Corollary}\newcommand{\bt}{\begin{theorem}\em}
\newcommand{\et}{\end{theorem}}
%\newcommand{\qed}{$\Box$}
\newcommand{\Proof}{{\noindent\bf Proof. }}
\DeclareMathOperator*{\argmax}{arg\,max}
\DeclareMathOperator*{\argmin}{arg\,min}

\def\etal{\textit{et al.}}
\def\ie{\textit{i.e.}}
\def\eg{\textit{e.g.}}

\newcommand{\nin}{\noindent}

\setlength{\intextsep}{10pt plus 2pt minus 2pt}

\newcommand{\bea}{\begin{eqnarray}}
\newcommand{\eea}{\end{eqnarray}}

\newcommand{\bdf}{\begin{df}\em}
\newcommand{\edf}{\end{df}}

\newcommand{\ben}{\begin{enumerate}}
\newcommand{\een}{\end{enumerate}}

\newcommand{\dist}{\operatorname{dist}}

\newcommand{\avg}{\operatorname{avg}}

\definecolor{grey}{rgb}{0.4,0.4,0.4}
\definecolor{loselots}{rgb}{1,0.4,0.4}
\definecolor{losesome}{rgb}{1,0.6,0.6}
\definecolor{losebit}{rgb}{1,0.8,0.8}
\definecolor{tie}{rgb}{1,1,1}
\definecolor{winbit}{rgb}{0.8,1,0.8}
\definecolor{winsome}{rgb}{0.6,1,0.6}
\definecolor{winlots}{rgb}{0.4,1,0.4}


%\icmltitle{No-Regret Learning in Extensive-Form Games with Imperfect Recall}
\title{Monte Carlo Tree Search with Heuristic Evaluations\\using Implicit Minimax Backups}

\author{Marc Lanctot$^1$, Mark H.M. Winands$^1$, Tom Pepels$^1$, Nathan R. Sturtevant$^2$ \\
\hspace{-1cm}$^1$ Department of Knowledge Engineering, Maastricht University\\
\hspace{-1cm}$^2$ Computer Science Department, University of Denver\\}

\date{}

% The \icmltitle you define below is probably too long as a header.
% Therefore, a short form for the running title is supplied here:
%\icmltitlerunning{No-Regret Learning in Extensive-Form Games with Imperfect Recall}

\newif\iftechreport
\techreporttrue

\begin{document} 

% It is OKAY to include author information, even for blind
% submissions: the style file will automatically remove it for you
% unless you've provided the [accepted] option to the icml2012
% package.
%\icmlauthor{Your Name}{email@yourdomain.edu}
%\icmladdress{Your Fantastic Institute,
%            314159 Pi St., Palo Alto, CA 94306 USA}
%\icmlauthor{Your CoAuthor's Name}{email@coauthordomain.edu}
%\icmladdress{Their Fantastic Institute,
%            27182 Exp St., Toronto, ON M6H 2T1 CANADA}

\maketitle
\allowdisplaybreaks
% You may provide any keywords that you 
% find helpful for describing your paper; these are used to populate 
% the "keywords" metadata in the PDF but will not be shown in the document
%\icmlkeywords{boring formatting information, machine learning, ICML}

\begin{abstract} 
Monte Carlo Tree Search (MCTS) has improved the performance of game-playing engines in 
domains such as Go, Hex, and general-game playing. MCTS has been shown to outperform
outperform classic alpha-beta search in games where good heuristic evaluations are difficult to obtain. 
In recent years, combining ideas from traditional minimax search in MCTS has been shown to be advantageous in some domains, 
such as Lines of Action, Amazons, and Breakthrough.
In this paper, we propose 
a new way to use heuristic evaluations to guide the MCTS search by storing the two sources of 
information, estimated win rates and heuristic evaluations, separately. 
Rather than using the heuristic evaluations to replace the playouts, 
our technique backs them up {\it implicitly} during its MCTS simulations. 
These learned evaluation values are then used to guide future simulations. 
Compared to current techniques, we show that using implicit minimax backups  
leads to stronger play performance in Breakthrough, Lines of Action, and Kalah. 
\end{abstract} 

%
%%% BEGIN PAPER CONTENT
%

%
%%% END PAPER CONTENT
%


\bibliography{im-mcts}
\bibliographystyle{plainnat}

\appendix

\section{Appendix A: Game Images}

\section{Appendix B: Tournaments and Playout Comparisons}

This appendix includes details of the results of played games to determine the best baseline players. 

\subsection{Parameter Values for Breakthrough and Kalah}

\begin{table}[h!]
\begin{center}
\begin{tabular}{|l|l|}
\hline
Technique & Parameter set \\
\hline
fet$x$          & $\{ 0, 1, \ldots, 5, 8, 10, 12, 16, 20, 30, 50, 100, 1000 \}$ \\
det$x$         & $\{ .1, .2, .3, , .4, .5, .55, .6, .65, .7, .75, .8, .85, .9 \}$ \\
ege$\epsilon$  & $\{ 0, .05, .1, .15, .2, .3, .4, .5, .6, .7, .8, .9, 1 \}$ \\
im$\alpha$     & $\{ 0, .05, .1, .15, \ldots, .55, .6, .75, 1 \}$ \\
\hline
\end{tabular}
\end{center}
\caption{Parameter value sets.}
\label{tbl:parmsets}
\end{table}

\subsection{Kalah Playout Optimization}

{\bf {\color{red} Missing: epsilon-greedy and comparisons to fet.}}

\subsubsection{Fixed Early Termination Tournament}

Each matchup included 1000 games, but only the wins and losses are removed. The remaining
matches were draws. 

\begin{verbatim}
round 1

winner mcts_h_fet0 (368) vs. loser mcts_h_fet1000 (61)
winner mcts_h_fet1 (408) vs. loser mcts_h_fet100 (61)
winner mcts_h_fet2 (458) vs. loser mcts_h_fet50 (61)
winner mcts_h_fet3 (460) vs. loser mcts_h_fet30 (37)
winner mcts_h_fet4 (429) vs. loser mcts_h_fet20 (44)
winner mcts_h_fet5 (223) vs. loser mcts_h_fet10 (83)
mcts_h_fet8 gets a by

round 2

winner mcts_h_fet0 (181) vs. loser mcts_h_fet8 (169)
winner mcts_h_fet5 (189) vs. loser mcts_h_fet1 (116)
winner mcts_h_fet4 (166) vs. loser mcts_h_fet2 (115)
mcts_h_fet3 gets a by

round 3

winner mcts_h_fet3 (161) vs. loser mcts_h_fet0 (124)
winner mcts_h_fet4 (132) vs. loser mcts_h_fet5 (122)

round 4

winner mcts_h_fet4 (139) vs. loser mcts_h_fet3 (110)

Winner: mcts_h_fet4
\end{verbatim}


\subsection{Breakthrough Playout Enhancement Optimization (using efMS evaluator)}

{\bf {\color{red} Missing: det tuning.}}

\subsubsection{Fixed Early Terminations Tournament}

\begin{verbatim}
round 1

winner mcts_h_fet1000 (115) vs. loser mcts_h_fet0 (85)
winner mcts_h_fet100 (117) vs. loser mcts_h_fet1 (83)
winner mcts_h_fet50 (108) vs. loser mcts_h_fet2 (92)
winner mcts_h_fet30 (138) vs. loser mcts_h_fet3 (62)
winner mcts_h_fet20 (129) vs. loser mcts_h_fet4 (71)
winner mcts_h_fet10 (129) vs. loser mcts_h_fet5 (71)
mcts_h_fet8 gets a by

round 2

winner mcts_h_fet8 (108) vs. loser mcts_h_fet1000 (92)
winner mcts_h_fet10 (112) vs. loser mcts_h_fet100 (88)
winner mcts_h_fet20 (128) vs. loser mcts_h_fet50 (72)
mcts_h_fet30 gets a by

round 3

winner mcts_h_fet30 (113) vs. loser mcts_h_fet8 (87)
winner mcts_h_fet20 (104) vs. loser mcts_h_fet10 (96)

round 4

winner mcts_h_fet20 (104) vs. loser mcts_h_fet30 (96)

Winner: mcts_h_fet20
\end{verbatim}

\subsubsection{Epsilon-greedy Playout Tournament}

\begin{verbatim}
round 1

winner mcts_h_ege0.0 (156) vs. loser mcts_h_ege1.0 (44)
winner mcts_h_ege0.05 (155) vs. loser mcts_h_ege0.9 (45)
winner mcts_h_ege0.1 (156) vs. loser mcts_h_ege0.8 (44)
winner mcts_h_ege0.15 (153) vs. loser mcts_h_ege0.7 (47)
winner mcts_h_ege0.2 (151) vs. loser mcts_h_ege0.6 (49)
winner mcts_h_ege0.3 (119) vs. loser mcts_h_ege0.5 (81)
mcts_h_ege0.4 gets a by

round 2

winner mcts_h_ege0.0 (115) vs. loser mcts_h_ege0.4 (85)
winner mcts_h_ege0.05 (119) vs. loser mcts_h_ege0.3 (81)
winner mcts_h_ege0.1 (125) vs. loser mcts_h_ege0.2 (75)
mcts_h_ege0.15 gets a by

round 3

winner mcts_h_ege0.15 (103) vs. loser mcts_h_ege0.0 (97)
winner mcts_h_ege0.1 (110) vs. loser mcts_h_ege0.05 (90)

round 4

winner mcts_h_ege0.1 (108) vs. loser mcts_h_ege0.15 (92)

Winner: mcts_h_ege0.1
\end{verbatim}

\subsubsection{Tournament Winner Comparisons}

\begin{table}[h!]
\begin{center}
\begin{tabular}{|c|c|ccc|}
\hline
Player A & Player B                             & A Wins (\%)  & B Wins (\%)  & Ties \\ 
\hline
MCTS(ege$0.1$,det$0.5$) & MCTS(ege$0.1$)        & 738 (78.2)   & 262 (26.2)   & 0    \\
MCTS(ege$0.1$,det$0.5$) & MCTS(fet$20$,det$0.5$) & 633 (63.3)   & 367 (36.7)   & 0    \\
MCTS(ege$0.1$)          & MCTS(fet$20$)          & 557 (55.7)   & 443 (44.3)   & 0    \\
MCTS(ege$0.1$)          & MCTS(fet$4$)           & 768 (76.8)   & 232 (23.2)   & 0    \\
\hline
\end{tabular}
\end{center}
\caption{Breakthrough playout comparisons.}
\end{table}

\subsection{Breakthrough Playout Enhancement Optimization (using efLH evaluator)}

\subsubsection{Fixed Early Terminations Tournament}

\begin{verbatim}
round 1

winner mcts_h_efv1_fet0 (118) vs. loser mcts_h_efv1_fet1000 (82)
winner mcts_h_efv1_fet1 (129) vs. loser mcts_h_efv1_fet100 (71)
winner mcts_h_efv1_fet2 (113) vs. loser mcts_h_efv1_fet50 (87)
winner mcts_h_efv1_fet3 (101) vs. loser mcts_h_efv1_fet30 (99)
winner mcts_h_efv1_fet20 (121) vs. loser mcts_h_efv1_fet4 (79)
winner mcts_h_efv1_fet5 (100) vs. loser mcts_h_efv1_fet16 (100)
winner mcts_h_efv1_fet8 (102) vs. loser mcts_h_efv1_fet12 (98)
mcts_h_efv1_fet10 gets a by

round 2

winner mcts_h_efv1_fet10 (108) vs. loser mcts_h_efv1_fet0 (92)
winner mcts_h_efv1_fet8 (112) vs. loser mcts_h_efv1_fet1 (88)
winner mcts_h_efv1_fet5 (115) vs. loser mcts_h_efv1_fet2 (85)
winner mcts_h_efv1_fet20 (123) vs. loser mcts_h_efv1_fet3 (77)

round 3

winner mcts_h_efv1_fet20 (110) vs. loser mcts_h_efv1_fet10 (90)
winner mcts_h_efv1_fet8 (101) vs. loser mcts_h_efv1_fet5 (99)

round 4

winner mcts_h_efv1_fet8 (106) vs. loser mcts_h_efv1_fet20 (94)

Winner: mcts_h_efv1_fet8
\end{verbatim}

\subsubsection{Epsilon-greedy Playout Tournament}

\begin{verbatim}
round 1

winner mcts_h_efv1_ege1.0 (136) vs. loser mcts_h_efv1_ege0.0 (64)
winner mcts_h_efv1_ege0.9 (121) vs. loser mcts_h_efv1_ege0.05 (79)
winner mcts_h_efv1_ege0.1 (110) vs. loser mcts_h_efv1_ege0.8 (90)
winner mcts_h_efv1_ege0.7 (103) vs. loser mcts_h_efv1_ege0.15 (97)
winner mcts_h_efv1_ege0.6 (104) vs. loser mcts_h_efv1_ege0.2 (96)
winner mcts_h_efv1_ege0.3 (100) vs. loser mcts_h_efv1_ege0.5 (100)
mcts_h_efv1_ege0.4 gets a by

round 2

winner mcts_h_efv1_ege1.0 (122) vs. loser mcts_h_efv1_ege0.4 (78)
winner mcts_h_efv1_ege0.3 (101) vs. loser mcts_h_efv1_ege0.9 (99)
winner mcts_h_efv1_ege0.6 (116) vs. loser mcts_h_efv1_ege0.1 (84)
mcts_h_efv1_ege0.7 gets a by

round 3

winner mcts_h_efv1_ege0.7 (102) vs. loser mcts_h_efv1_ege1.0 (98)
winner mcts_h_efv1_ege0.3 (105) vs. loser mcts_h_efv1_ege0.6 (95)

round 4

winner mcts_h_efv1_ege0.3 (110) vs. loser mcts_h_efv1_ege0.7 (90)

Winner: mcts_h_efv1_ege0.3
\end{verbatim}

\subsection{Tournament Winner Comparisons}

%    mcts_s_h_efv1_pd20 vs.                        mcts_s_h_efv1_pd8:   514   486     0 (diff    28, games  1000) 51.40 48.60 +/- 3.10
%  mcts_s_h_efv1_ege0.3 vs.                       mcts_s_h_efv1_pd20:   340   660     0 (diff  -320, games  1000) 34.00 66.00 +/- 2.94
%  mcts_s_h_efv1_ege0.3 vs.                     mcts_s_h_efv1_ege0.7:   510   490     0 (diff    20, games  1000) 51.00 49.00 +/- 3.10
%  mcts_s_h_efv1_ege0.7 vs.                       mcts_s_h_efv1_pd20:   194   806     0 (diff  -612, games  1000) 19.40 80.60 +/- 2.45
%  mcts_s_h_efv1_ege0.3 vs.                        mcts_s_h_efv1_pd8:   354   646     0 (diff  -292, games  1000) 35.40 64.60 +/- 2.96
%  mcts_s_h_efv1_ege0.7 vs.                        mcts_s_h_efv1_pd8:   255   745     0 (diff  -490, games  1000) 25.50 74.50 +/- 2.70

{\bf {\color{red} Missing: comparisons with det added.}}

\begin{table}[h!]
\begin{center}
\begin{tabular}{|c|c|ccc|}
\hline
Player A & Player B                  & A Wins (\%)  & B Wins (\%)  & Ties \\ 
\hline
MCTS(fet$8$)   & MCTS(fet$20$)       & 514 (51.4)   & 486 (48.6)   & 0    \\
MCTS(ege$0.3$) & MCTS(fet$0.7$)      & 510 (51.0)   & 490 (49.0)   & 0    \\
\hline
MCTS(ege$0.3$) & MCTS(fet$8$)        & 354 (35.4)   & 646 (64.6)   & 0    \\
MCTS(ege$0.3$) & MCTS(fet$20$)       & 340 (34.0)   & 660 (66.0)   & 0    \\
MCTS(ege$0.7$) & MCTS(fet$8$)        & 255 (25.5)   & 745 (74.5)   & 0    \\
MCTS(ege$0.7$) & MCTS(fet$20$)       & 194 (19.4)   & 806 (80.6)   & 0    \\
\hline
\end{tabular}
\end{center}
\caption{Breakthrough playout comparisons.}
\end{table}




\end{document} 
